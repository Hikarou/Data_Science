\documentclass[a4paper,10pt]{article}
\usepackage[utf8]{inputenc}			%encodage
\usepackage[T1]{fontenc}
\usepackage[french]{babel}
				%Mise en page
%\usepackage{layout}				%Affichage du gabarit de la page
\usepackage[top=0.5 in, bottom=0.5 in, left=0.5 in, right=0.5 in]{geometry}				%Modification des marges
\usepackage{geometry}				%Modification des marges
%\usepackage{setspace}				%Interligne
%\usepackage{soul}				%Barrer le texte
%\usepackage{ulem}				%Sous-ligner
%\usepackage{eurosym}				%Signe euro
%\usepackage{fancyhdr}
%\pagestyle{fancy}
%\usepackage{geometry}
%\geometry{top=3cm, bottom=3cm, left=3cm , right=0.5cm}
				%Pack de polices
%\usepackage{bookman}                    
%\usepackage{charter}
%\usepackage{newcent}
%\usepackage{lmodern}
%\usepackage{mathpazo}
%\usepackage{mathptmx}
				%Suite
%\usepackage{url}				%Cr\'{e}ation d'url
\usepackage{verbatim}				%Citation de code
%\usepackage{moreverb}				%Citation de code
\usepackage{listings}				%Citation de code color\'{e}
%\usepackage{fancyhdr}				%En tete et pied de page personnalis\'{e}
%\usepackage{multicol}				%Pouvoir ecrire sur plusieures colonnes
				%Figures
\usepackage{graphicx}				%Travailler avec des images
%\usepackage{wrapfig}				%Seulement en dernier recours
				%Couleur
%\usepackage{color}				%Colorer le texte 

%\usepackage{colortbl}				%Colorer les tableaux
				%Maths
%\usepackage{amsthm}				%Th\'{e}orèmes
\usepackage{amsmath}
%\usepackage{amssymb}
%\usepackage{mathrsfs}
				%Cr\'{e}ation d'index
%\usepackage{makeidx}
				%Creation d'arbres
%\usepackage{tikz}
\usepackage{enumitem}     %listings alphanum\'{e}iques
\usepackage{hyperref}
%\usepackage{minted}

% Title Page
\title{Projet Noté\\
\large\emph{Algorithmes Avancés}}
\author{José Ferro Pinto\and Fabrice Ceresa}
%\setcounter{section}{-1}

\begin{document}
\maketitle

\section{Partie 1}
%TODO
\section{Partie 2}
    Pour normaliser les données, nous avons utilisé \verb?preprocessing.normalize(raw['data'])? comme demandé dans le TP.
    
    Le programme se passe en 4 boucles intriquées :
    \begin{enumerate}
        \item Pour chacun des datasets
        \item Pour chacun des classificateurs
        \item Pour chacun des paramètres
        \item Pour chacune des étapes de la validation croisée
    \end{enumerate}
    
    Les datasets utilisés : 
    \begin{itemize}
        \item \verb?load_breast_cancer? pour les données du cancer du sein
        \item \verb?load_wine? pour les données sur du vin
    \end{itemize}
    
    Les différents classificateurs utilisés et leurs paramètres:
    \begin{itemize}
        \item \verb?KNeighborsClassifier? pour la méthode des K-plus proches voisins.
        
        Paramètre variable : \verb?n_neighbors? nombre de voisins pris en compte : [1,51] avec un pas de 5
        \item \verb?DecisionTreeClassifier? pour les arbres de décisions. 
        
        Paramètre variable: \verb?min_samples_leaf? Nombre d'objets à partir duquel on comptabilise une feuille [2, 52] avec un pas de 5
        \item \verb?MLPClassifier? pour le Perceptron multi-couche avec une couche utilisant ``stochastic gradient descent''
        
        Paramètres fixes : \verb?solver='sgd', activation='logistic', max_iter=1000, verbose=False, learning_rate_init=0.1, tol=0., early_stopping=True?
        
        Paramètre variable : \verb?hidden_layer_sizes=(nodes,)? donc une couche et nodes varie entre 2 et 20 par pas de 3.
        \item \verb?MLPClassifier? pour le Perceptron multi-couche avec une couche utilisant ``a stochastic gradient-based optimizer''
        
        Paramètres fixes : \verb?solver='adam', activation='logistic', max_iter=1000, verbose=False, learning_rate_init=0.1, tol=0., early_stopping=True?
        
        Paramètre variable : \verb?hidden_layer_sizes=(nodes,)? donc une couche et nodes varie entre 2 et 20 par pas de 3.
        \item \verb?MLPClassifier? pour le Perceptron multi-couche avec deux couches utilisant ``stochastic gradient descent''
        
        Paramètres fixes : \verb?solver='sgd', activation='logistic', max_iter=1000, verbose=False, learning_rate_init=0.1, tol=0., early_stopping=True?
        
        Paramètre variable : \verb?hidden_layer_sizes=(nodes,)? donc une couche et nodes varie entre 2 et 20 par pas de 3.
        \item \verb?MLPClassifier? pour le Perceptron multi-couche avec deux couches utilisant ``a stochastic gradient-based optimizer''
        
        Paramètres fixes : \verb?solver='adam', activation='logistic', max_iter=1000, verbose=False, learning_rate_init=0.1, tol=0., early_stopping=True?
        
        Paramètre variable : \verb?hidden_layer_sizes=(nodes,)? donc une couche et nodes varie entre 2 et 20 par pas de 3.  
    \end{itemize}
    
    La validation est donc la validation croisée : \verb?RepeatedKFold?
    
    Nous avons décidé de prendre les mesures suivantes :
    \begin{itemize}
        \item le temps d'exécution
        \item la moyenne des scores
        \item l'écart type des scores
    \end{itemize}
    
    La sortie de l'exécution du script python donne la sortie suivante : 
    
    \verbatiminput{partie2.txt}


\section{Partie 3}

\end{document}
